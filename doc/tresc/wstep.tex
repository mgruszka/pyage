\chapter{Wprowadzenie}
\label{cha:wstep}


Optymalizacja jest wyznaczeniem spośród dopuszczalnych rozwiązań danego problemu, rozwiązania najlepszego ze względu na przyjęte kryterium (wskaźnik) jakości (np. koszt, zysk, niezawodność). Optymalizacja ciągła jest takim rodzajem optymalizacji, w którym wszystkie zmienne funkcji celu są ciągłe, czyli należą do nieprzerwanego zbioru liczb. Często problemy optymalizacyjne są problemami, w których przestrzeń przeszukiwań jest za duża lub zbyt skomplikowana, żeby można było ją eksplorować jednocześnie wydajnie i dokładnie.

Przy rozwiązywaniu tak złożonych problemów wykorzystywane są heurystyki. Heurystyką nazywamy taką metodę znajdowania rozwiązań, która nie daje gwarancji znalezienia optymalnego rozwiązania. Brak gwarancji znalezienia rozwiązania wynika z faktu, że heurystyki przeszukują przestrzeń rozwiązań i wraz z postępem algorytmu modyfikują jego składowe parametry dostosowywując się do osiąganych wyników pośrednich.

Algorytm roju cząstek, jak również algorytm ewolucyjny wykorzystywany w niniejszej pracy w celu porównania skuteczności są heurystykami. Doskonalą one rozwiązanie danego problemu poprzez iteracyjne modyfikowanie zmiennych decyzyjnych. Algorytmy ewolucyjne czerpią z obserwowanego w przyrodzie procesu ewolucji - podczas działania algorytmu jego najlepiej przystosowane składowe są iteracyjnie modyfikowane, natomiast te przystosowane najgorzej są odrzucane. Algorytm roju cząstek bazuje na zachowaniach stadnych. W każdej kolejnej iteracji elementy algorytmu wpływają na siebie na podstawie swojego położenia w przestrzeni rozwiązań.



%\section{Cele pracy}
%\label{sec:celePracy}
Tematem niniejszej pracy jest sprawdzenie skuteczności algorytmu roju cząstek w wybranych problemach optymalizacji ciągłej. Skuteczność rozwiązania została porównana z istniejącym rozwiązaniem zaimplementowanym na platformie pyAgE. Celem pracy będzie implementacja algorytmu roju cząstek oraz zbadanie jego skuteczności w wybranych problemach benchmarkowych. W implementacji zostanie wykorzystane istniejące środowisko agentowe - pyAgE. Wspiera ono budowę rozproszonych modeli obliczeniowych oraz dostarcza zestaw narzędzi umożliwiających porównanie zaimplementowanych rozwiązań.

Finalnym efektem będzie porównanie otrzymanych rozwiązań z istniejącym algorytmem ewolucyjnym - EMAS. Aby to osiągnąć zostaną przeprowadzone analizy składowych parametrów wykonanych modyfikacji algorytmu rojowego, a następnie zostaną wykorzystane te, które dadzą najdokładniejsze rozwiązanie.


%\section{Zakres pracy}
W czasie realizacji pracy, zostaną stworzone komponenty na platformę pyAgE dające możliwość dokonywania obliczeń za pomocą algorytmu roju cząstek. Zostaną zaimplementowane zarówno elementy składowe algorytmu, jak i przetestowane konfiguracje pozwalające na badanie jego skuteczności.  Porównanie odbędzie się na platformie pyAgE pomiędzy istniejącym algorytmem ewolucyjnym EMAS, a różnymi konfiguracjami algorytmu roju cząstek. Algorytm roju cząstek zostanie przystosowany, aby sprawdzić jego skuteczność w kilku wersjach. Podstawowej wersji, zawierającej możliwość sterowania prędkością oraz wagami składowych. Wersji rozszerzonej o kroki algorytmu ewolucyjnego wykonywane razem z rojowymi. Rozszerzonej o informację o sąsiadach, dostarczaną przez platformę pyAgE oraz wzbogaconą o wyspy obliczeniowe.

%\section{Zawartość pracy}
Początek pracy, rozdział \ref{cha:genetyczne}, zawieta informacje o technikach ewolucyjnych. Następnie w rozdziale \ref{cha:pso} znajduje się opis działania systemów rojowych. Kolejny rozdział, \ref{cha:ewaluacja}, opisuje sposób zbierania i porównywania otrzymanych wyników. Znajduje się tam opis wybranych aspektów implementacyjnych jak również informacje o funkcji benchamrkowej stosowanej do porównania wyników niniejszej pracy. W rozdziale \ref{cha:psoTesty} znajdują się informacje o zaimplementowanych rozwiązaniach oraz wyniki badań prowadzące do uzyskania najlepszych parametrów algorytmu roju cząstek. Zakończenie pracy poświęcone zostało porównaniu algorytmów rojowego i ewolucyjnego oraz analiza skuteczności algorytmu rojowego pod kątem postawionego problemu.



