\chapter{Wprowadzenie}
\label{cha:wstep}


Tematem niniejszej pracy jest sprawdzenie skuteczności algorytmu roju cząstek w wybranych problemach optymalizacji ciągłej. Skuteczność rozwiązania została oceniona względem dostarczonej platformy wraz z istniejącym rozwiązaniem.

Algorytm roju cząstek jest heurystyką, która doskonali rozwiązanie danego problemu, poprzez iteracyjne modyfikowanie parametrów. Dokładny opis działania algorytmów rojowych został opisany w rozdziale \ref{cha:pso}.

Heurystyką nazywamy taką metodę znajdowania rozwiązań, która nie daje gwarancji znalezienia optymalnego rozwiązania. Metody takie używane są w przypadkach, gdy pełny algorytm jest nieznany lub jest zbyt kosztowny w wykonaniu.

Wykorzystywane w niniejszej pracy w celu porównania skuteczności algorytmy ewolucyjne, również należą do heurystyk. Ich metoda działania została opisana w rozdziale \ref{cha:genetyczne}.

Optymalizacja jest to wyznaczenie spośród dopuszczalnych rozwiązań danego problemu rozwiązania najlepszego za względu na przyjęte kryterium (wskaźnik) jakości (np. koszt, zysk, niezawodność). Optymalizacja ciągła jest takim rodzajem optymalizacji, w którym wszystkie zmienne funkcji celu są ciągłe, czyli należa do nieprzerwanego zbioru liczb.


\section{Cele pracy}
\label{sec:celePracy}
Celem pracy jest realizacja implementacji algorytmu roju cząstek, oraz badanie ich skuteczności. W implementacji zostanie wykorzystane istniejące środowisko agentowe (pyAgE). Wspiera ono budowę rozproszonych modeli obliczeniowych oraz zostało napisane w języku Python.

Dzięki zastosowaniu środowiska agentowego zostanie sprawdzona użyteczność algorytmów rojowych przy przyjęciu każdego osobnika jako osobnego agenta.

Skuteczność zaimplementowanych rozwiązań będzie sprawdzana w wybranych problemach benchmarkowych. Warianty konfiguracji jak i dobrane parametry zostaną dobrane eksperymentalnie. 

Finalnym celem jest porównanie otrzymanych rozwiązań z już istniejącym na platformie algorytmem ewolucyjnym - EMAS.


\section{Zakres pracy}
W czasie realizacji pracy, została rozszerzona platforma pyAgE o możliwość dokonywania obliczeń za pomocą algorytmu roju cząstek. Zostały zaimplementowane zarówno elementy składowe algorytmu jak i przetestowane konfiguracje pozwalające na badanie skuteczności algorytmu. 

Porównanie odbywało się pomiędzy dostarczonym wraz z platformą pyAgE algorytmem genetycznym EMAS, a różnymi konfiguracjami algorytmu roju cząstek. 

Algorytm roju cząstek został przystosowany aby sprawdzić jego skuteczność w kilku wersjach:
\begin{itemize}
\item Podstawowa wersja, zawierająca możliwość sterowania:
\begin{itemize}
\item Prędkością
\item Wagami składowych
\end{itemize}
\item Rozszerzoną o kroki algorytmu ewolucyjnego wykonywane razem z rojowymi
\item Rozszerzoną o informację o sąsiadach, dostarczaną przez platformę pyAgE
\item Rozszerzoną o wyspy obliczeniowe
\end{itemize}


\section{Zawartość pracy}
Rozdział \ref{cha:pyage} zawieraja informację o platformie pyAgE i jej komponentach. Znajdują się tam również informacje o funkcjach benchamrkowych, ze szczególnym uwzględnieniem funkcji stosowanej do porównania w niniejszej pracy.

Następną częscią pracy (rozdziały \ref{cha:pso} i \ref{cha:genetyczne}) jest przybliżenie mechanizmów działania zarówno algorytmów rojowych jak i ewolucyjnych, szczególnie tych, które zostały zastosowane w implementacji. W rodziale \ref{sec:psoModyfikacje} zostały opisane częste modyfikacje algorytmu roju cząstek.

W rozdziale \ref{cha:psoTesty} znajdują się informację o zaimplementowanych rozwiązaniach oraz wyniki badań prowadzące do uzyskania najlepszych parametrów algorytmu roju cząstek.

Pod koniec pracy znajdują się porównania algorytmów rojowego i ewolucyjnego oraz analiza skuteczności algorytmu rojowego pod kątem postawionego problemu.











