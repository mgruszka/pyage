\chapter{Wprowadzenie}
\label{cha:wstep}


Tematem niniejszej pracy jest sprawdzenie skuteczności algorytmu roju cząstek w wybranych problemach optymalizacji ciągłej. Skuteczność rozwiązania została oceniona względem dostarczonej platformy wraz z istniejącym rozwiązaniem.


\section{Cele pracy}
\label{sec:celePracy}
Celem pracy jest realizacja różnych wariantów implementacji algorytmu roju cząstek oraz badanie ich skuteczności w wybranych problemach optymalizacji ciągłej. W implementacji wykorzystane zostanie istniejące środowisko wspierające budowę rozproszonych modeli obliczeniowych w języku Python (pyAgE). Skuteczność algorytmu roju cząstek badana będzie eksperymentalnie dla różnych wariantów konfiguracji obliczeń oraz różnych problemów benchmarkowych.

\section{Zakres pracy}
W czasie realizacji pracy, została rozszerzona platforma pyAgE o możliwość dokonywania obliczeń za pomocą algorytmu roju cząstek. Zostały zaimplementowane zarówno elementy składowe algorytmu jak i przetestowane konfiguracje pozwalające na badanie skuteczności algorytmu. 

Porównanie odbywało się pomiędzy dostarczonym wraz z platformą pyAgE algorytmem genetycznym EMAS, a różnymi konfiguracjami algorytmu roju cząstek. 

Algorytm roju cząstek został przystosowany aby sprawdzić jego skuteczność w kilku wersjach:
\begin{itemize}
\item Podstawowa wersja, zawierająca możliwość sterowania:
\begin{itemize}
\item Prędkością
\item Wagami składowych
\end{itemize}
\item Rozszerzoną o kroki algorytmu ewolucyjnego wykonywane razem z rojowymi
\item Rozszerzoną o informację o sąsiadach, dostarczaną przez platformę pyAgE
\item Rozszerzoną o wyspy obliczeniowe
\end{itemize}


\section{Zawartość pracy}
Rozdział \ref{cha:pyage} zawieraja informację o platformie pyAgE i jej komponentach. Znajdują się tam również informacje o funkcjach benchamrkowych, ze szczególnym uwzględnieniem funkcji stosowanej do porównania w niniejszej pracy.

Następną częscią pracy(rozdziały \ref{cha:pso} i \ref{cha:genetyczne}) jest przybliżenie mechanizmów działania zarówno algorytmów rojowych jak i ewolucyjnych, szczególnie tych, które zostały zastosowane w implementacji. W rodziale \ref{sec:psoModyfikacje} zostały też częste modyfikacje algorytmu roju cząstek.

Rozdział \ref{cha:psoTesty} zawiera informację o zaimplementowanych rozwiązaniach oraz wyniki badań prowadzące do uzyskania najlepszych parametrów algorytmu roju cząstek.

Pod koniec pracy znajdują się porównania algorytmów rojowego i ewolucyjnego oraz analiza skuteczności algorytmu rojowego pod kątem postawionego problemu.


\chapter{Rozdział o pyagu i reszcie}
\label{cha:pyage}

\section{Platforma PyAGE}



\section{Sposób porównywania wyników}
\label{sec:porownywanieWynikow}





\subsection{Funkcja Rastrigina}
\label{sec:rastrigin}















